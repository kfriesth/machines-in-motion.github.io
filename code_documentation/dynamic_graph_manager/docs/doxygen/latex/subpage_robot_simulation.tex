A robot implemented with dynamic graph manager doesn\textquotesingle{}t come with a simulator by default.

There are multiple ways and at different levels to implement a simulator\+:
\begin{DoxyItemize}
\item Implement a simulator in C++ and start it as dynamic graph manager from the console
\item Create a device in python and use another simulator from python
\end{DoxyItemize}\hypertarget{subpage_robot_simulation_simu_sec_cpp}{}\section{7.\+1/ Simulator implementation in C++}\label{subpage_robot_simulation_simu_sec_cpp}
For an example, see \href{https://git-amd.tuebingen.mpg.de/amd-clmc/dynamic_graph_manager/blob/master/demos/single_motor/single_motor_device_simulator.cpp}{\tt https\+://git-\/amd.\+tuebingen.\+mpg.\+de/amd-\/clmc/dynamic\+\_\+graph\+\_\+manager/blob/master/demos/single\+\_\+motor/single\+\_\+motor\+\_\+device\+\_\+simulator.\+cpp}.\hypertarget{subpage_robot_simulation_simu_sec_python}{}\section{7.\+2/ Device in python}\label{subpage_robot_simulation_simu_sec_python}
To test a dynamic graph implementation, one way is to create a Device object in python. This device object is created using the same / a similar robot-\/yaml file as for the dynamic graph manger\textquotesingle{}s real robot. At every integration step of the simulation, the sensor values must be stored on the sensor signals of the device and the dynamic graph is executed. Last, the control commands must be forwarded to a simulator, the simulator is stepped and the next cycle begins.

The following code shows how to create a device and robot. It also scatches how to do the simulation loop. 
\begin{DoxyCode}
from \hyperlink{namespacedynamic__graph__manager}{dynamic\_graph\_manager}.device \textcolor{keyword}{import} Device
from \hyperlink{namespacedynamic__graph__manager}{dynamic\_graph\_manager}.device.robot \textcolor{keyword}{import} Robot

\textcolor{preprocessor}{# Create the device and robot object.}
device = Device(\textcolor{stringliteral}{"your\_robot\_name"})
device.initialize("path\_to\_the\_robot\_yaml\_file.yaml")
\hyperlink{namespacerobot}{robot} = Robot(name=local\_device.name, device=local\_device)

\textcolor{preprocessor}{# This creates the controllers and plugs the sensor to the control signals etc.}
setup\_controller(\hyperlink{namespacerobot}{robot})

\textcolor{preprocessor}{# Setup your simulator. This might be pybullet etc.}
sim = create\_simulator()

\textcolor{preprocessor}{# Do the simulation loop. Here, we assume to simulate 4000 steps}
\textcolor{keywordflow}{for} it in range(4000):
  # Fill all the sensor signals based on the simulator values.
  \hyperlink{namespacerobot}{robot}.device.<join\_positions>.value = sim.get\_joint\_positions()
  ...

  # Execute the dynamic graph. This computes the control output.
  \hyperlink{namespacerobot}{robot}.device.executeGraph()

  # Pass the computed control signals to the simulator.
  sim.control(\hyperlink{namespacerobot}{robot}.device.<ctrl\_commands>.value)

  # Step the simulator by one dt.
  sim.step()
\end{DoxyCode}
 