\hypertarget{subpage_basic_control_graph_control_sec_intro}{}\section{5.\+1/ Good practice}\label{subpage_basic_control_graph_control_sec_intro}
\begin{DoxyRefDesc}{Todo}
\item[\hyperlink{todo__todo000001}{Todo}]Create a pacckage like ci\+\_\+integration that depends on Dynamic Graph and show the implementation of the dg\+\_\+ci\+\_\+example.\end{DoxyRefDesc}



\begin{DoxyCode}
WARNING:

Please make sure to read the following page first: 4/ Implementing a \hyperlink{namespacerobot}{robot} in 
dynamic graph manager
\end{DoxyCode}


If possible, controllers and helper functions should be implemented in \char`\"{}dg\+\_\+tools\char`\"{}. Please refer to the project\textquotesingle{}s readme about the desired code structure.

\begin{DoxyRefDesc}{Todo}
\item[\hyperlink{todo__todo000002}{Todo}]cleanly split between a simulated example (currently shown with stuggihop) and a hardware example (currently at the bottom with a single motor...).\end{DoxyRefDesc}
\hypertarget{subpage_basic_control_graph_control_sec_simple_entity}{}\section{5.\+2/ Writing a simple entity}\label{subpage_basic_control_graph_control_sec_simple_entity}
In dynamic graph, an entity is an object that consumes input signals and provides output signals. For instance, a P-\/controller might consume the current position of a motor as input signal and provide as output a torque command to move the motor position towards a desired position.

To control a robot / simulator using dynamic graph manager, the following is necessary\+:
\begin{DoxyItemize}
\item Implement a controller as a dynamic graph entity.
\item Example\+:
\begin{DoxyItemize}
\item \href{https://git-amd.tuebingen.mpg.de/amd-clmc/dg_tools/blob/master/src/control/control_pd.cpp}{\tt https\+://git-\/amd.\+tuebingen.\+mpg.\+de/amd-\/clmc/dg\+\_\+tools/blob/master/src/control/control\+\_\+pd.\+cpp}
\item \href{https://git-amd.tuebingen.mpg.de/amd-clmc/dg_tools/blob/master/include/dg_tools/control/control_pd.hpp}{\tt https\+://git-\/amd.\+tuebingen.\+mpg.\+de/amd-\/clmc/dg\+\_\+tools/blob/master/include/dg\+\_\+tools/control/control\+\_\+pd.\+hpp}
\end{DoxyItemize}
\item Expose the entity using a few lines in C\+Make\+List.\+txt
\item Example\+:
\begin{DoxyItemize}
\item \href{https://git-amd.tuebingen.mpg.de/amd-clmc/dg_tools/blob/master/src/CMakeLists.txt}{\tt https\+://git-\/amd.\+tuebingen.\+mpg.\+de/amd-\/clmc/dg\+\_\+tools/blob/master/src/\+C\+Make\+Lists.\+txt}
\end{DoxyItemize}
\item In the python interpreter of the dynamic graph manager, create the control graph, connect it to the robot device and run configurations if needed
\item Example (T\+O\+DO\+: make an example contained in dg\+\_\+tools)\+:
\begin{DoxyItemize}
\item \href{https://git-amd.tuebingen.mpg.de/amd-clmc/dg_blmc_robots/blob/master/demos/stuggihop/simulations/dg_stuggihop_simu_basic.py}{\tt https\+://git-\/amd.\+tuebingen.\+mpg.\+de/amd-\/clmc/dg\+\_\+blmc\+\_\+robots/blob/master/demos/stuggihop/simulations/dg\+\_\+stuggihop\+\_\+simu\+\_\+basic.\+py}
\end{DoxyItemize}
\end{DoxyItemize}\hypertarget{subpage_basic_control_graph_control_sec_expose_entity}{}\section{5.\+3/ Expose the entities to Python}\label{subpage_basic_control_graph_control_sec_expose_entity}
Once you have defined the controller in C++, you need to expose it to python. This is done by adding a few lines into the C\+Make\+List.\+txt file.

For an example, please refer to \href{https://git-amd.tuebingen.mpg.de/amd-clmc/dg_tools/blob/master/src/CMakeLists.txt}{\tt https\+://git-\/amd.\+tuebingen.\+mpg.\+de/amd-\/clmc/dg\+\_\+tools/blob/master/src/\+C\+Make\+Lists.\+txt}.
\begin{DoxyItemize}
\item First add your entities here\+: 
\item Secondly export the python bindings\+: 
\end{DoxyItemize}


\begin{DoxyCode}
WARNING:

one need the following lines at the beginning of the CMakeLists.txt because 
otherwise the python bindings will not link to your library and will not find 
your entities.
\end{DoxyCode}
 \hypertarget{subpage_basic_control_graph_control_sec_load_graph}{}\section{5.\+4/ Create the control graph entries and connect them to the robot}\label{subpage_basic_control_graph_control_sec_load_graph}
An example on how a simple control graph using the Motor\+Controller with the robot.\+device is shown in single\+\_\+motor\+\_\+main.\+py. Note that {\ttfamily robot.\+device} is provided by the Dynamic Graph Manager automatically and initialized to talk to the real/simulated hardware. The robot.\+device is created from the yaml file\textquotesingle{}s device specification. Especially, all the input and output signals are created on the {\ttfamily robot.\+device} as described in the {\ttfamily sensor} and {\ttfamily control} part of the yaml file. 