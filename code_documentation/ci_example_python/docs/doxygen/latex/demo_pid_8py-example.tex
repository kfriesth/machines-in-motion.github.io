\hypertarget{demo_pid_8py-example}{}\section{demo\+\_\+pid.\+py}
In order to run this demos the {\ttfamily ci\+\_\+example\+\_\+python/python/ci\+\_\+example\+\_\+python} path needs to be added in the {\ttfamily P\+Y\+T\+H\+O\+N\+P\+A\+TH}. This is done after\+:
\begin{DoxyItemize}
\item 1. Building the workspace by executing {\ttfamily catkin\+\_\+make} from the root of the catkin workspace.
\item 2. \char`\"{}source ./devel/setup.\+bash\char`\"{} is called from the root of the catkin workspace.
\item 3. Run the demo by either\+:
\begin{DoxyItemize}
\item 3.\+1. {\ttfamily rosrun \hyperlink{namespaceci__example__python}{ci\+\_\+example\+\_\+python} \hyperlink{demo__pid_8py}{demo\+\_\+pid.\+py}}
\item 3.\+3. {\ttfamily cd /path/to/ci\+\_\+example\+\_\+python/} ; {\ttfamily ./demos/demo\+\_\+pid.py}
\end{DoxyItemize}
\end{DoxyItemize}

Notice the use of the double \#\# for the comments. This allow Doxygen to parse you code and for you to explain in details the content of the demo.


\begin{DoxyCodeInclude}
1 \textcolor{comment}{#!/usr/bin/env python}
2 
3 \textcolor{stringliteral}{""" @namespace Demos of the ci\_example\_python.pid.PID controller.}
4 \textcolor{stringliteral}{}
5 \textcolor{stringliteral}{@file demo\_pid.py}
6 \textcolor{stringliteral}{@copyright Copyright (c) 2017-2019,}
7 \textcolor{stringliteral}{           New York University and Max Planck Gesellschaft,}
8 \textcolor{stringliteral}{           License BSD-3-Clause}
9 \textcolor{stringliteral}{}
10 \textcolor{stringliteral}{@example demo\_pid.py}
11 \textcolor{stringliteral}{In order to run this demos the `ci\_example\_python/python/ci\_example\_python` path}
12 \textcolor{stringliteral}{needs to be added in the `PYTHONPATH`. This is done after:}
13 \textcolor{stringliteral}{- 1. Building the workspace by executing `catkin\_make` from the root of the}
14 \textcolor{stringliteral}{     catkin workspace.}
15 \textcolor{stringliteral}{- 2. "source ./devel/setup.bash" is called from the root of the catkin}
16 \textcolor{stringliteral}{     workspace.}
17 \textcolor{stringliteral}{- 3. Run the demo by either:}
18 \textcolor{stringliteral}{    - 3.1. `rosrun ci\_example\_python demo\_pid.py`}
19 \textcolor{stringliteral}{    - 3.3. `cd /path/to/ci\_example\_python/` ; `./demos/demo\_pid.py`}
20 \textcolor{stringliteral}{}
21 \textcolor{stringliteral}{Notice the use of the double ## for the comments. This allow Doxygen}
22 \textcolor{stringliteral}{to parse you code and for you to explain in details the content of the demo.}
23 \textcolor{stringliteral}{"""}
24 
25 
26 \textcolor{comment}{# Python 3 compatibility, has to be called just after the hashbang.}
27 \textcolor{keyword}{from} \_\_future\_\_ \textcolor{keyword}{import} print\_function, division
28 \textcolor{keyword}{from} \hyperlink{namespaceci__example__python_1_1pid}{ci\_example\_python.pid} \textcolor{keyword}{import} PID, get\_default\_pid, get\_config\_file\_pid
29 
30 
31 \textcolor{keywordflow}{if} \_\_name\_\_ == \textcolor{stringliteral}{"\_\_main\_\_"}:
32     
33     
34     current\_position = 1
35     
36     current\_velocity = 0.1
37     
38     position\_target = 0
39     
40     delta\_time = 0.001
41 
42     \textcolor{comment}{# basic example of PID usage}
43     print(\textcolor{stringliteral}{"basic pid usage:"})
44     
45     \textcolor{keyword}{class }Configuration:
46         \textcolor{stringliteral}{""" This a small shell that contains the PID gains.}
47 \textcolor{stringliteral}{        It mocks the load of a yaml files.}
48 \textcolor{stringliteral}{        Attributes:}
49 \textcolor{stringliteral}{            kp: Proportional gain.}
50 \textcolor{stringliteral}{            kd: Derivative gain.}
51 \textcolor{stringliteral}{            ki: Integral gain.}
52 \textcolor{stringliteral}{        """}
53         
54         \textcolor{keyword}{def }\_\_init\_\_(self, kp, kd, ki):
55             \textcolor{stringliteral}{"""  Create the 3 gains}
56 \textcolor{stringliteral}{            Args:}
57 \textcolor{stringliteral}{                kp: Proportional gain.}
58 \textcolor{stringliteral}{                kd: Derivative gain.}
59 \textcolor{stringliteral}{                ki: Integral gain.}
60 \textcolor{stringliteral}{            """}
61             self.kp = kp
62             self.kd = kd
63             self.ki = ki
64 
65     
66     config = Configuration(1,1,1)
67 
68     
69     pid = PID(config)
70     print(pid)
71     
72     force = pid.compute(current\_position,current\_velocity,position\_target,delta\_time)
73     print(\textcolor{stringliteral}{"force:"},force)
74 
75     
76     \textcolor{keywordflow}{print} (\textcolor{stringliteral}{"pid using default gains:"})
77     pid = \hyperlink{namespaceci__example__python_1_1pid_a6fc811a84ce986b8315aabae71987cbf}{get\_default\_pid}()
78     \textcolor{keywordflow}{print} (pid)
79 
80     force = pid.compute(current\_position,current\_velocity,position\_target,delta\_time)
81     \textcolor{keywordflow}{print} (\textcolor{stringliteral}{"force:"},force)
82 
83     \textcolor{comment}{# Example of creation of the PID using config files read from config file.}
84     \textcolor{keywordflow}{print} (\textcolor{stringliteral}{"pid using gains read from config file:"})
85     pid = \hyperlink{namespaceci__example__python_1_1pid_a18376955fadc78cf60e9836c120adc9a}{get\_config\_file\_pid}(verbose=\textcolor{keyword}{True})
86     \textcolor{keywordflow}{print} (pid)
87 
88     force = pid.compute(current\_position,current\_velocity,position\_target,delta\_time)
89     \textcolor{keywordflow}{print} (\textcolor{stringliteral}{"force:"},force)
\end{DoxyCodeInclude}
 