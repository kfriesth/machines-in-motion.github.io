\hypertarget{demo_ros_pid_8py-example}{}\section{demo\+\_\+ros\+\_\+pid.\+py}
In order to run this demos the {\ttfamily ci\+\_\+example\+\_\+python/python/ci\+\_\+example\+\_\+python} path needs to be added in the {\ttfamily P\+Y\+T\+H\+O\+N\+P\+A\+TH}. This is done after\+:
\begin{DoxyItemize}
\item 1. \char`\"{}catkin\+\_\+make\char`\"{} is called from the root of the catkin workspace
\item 2. \char`\"{}source ./devel/setup.\+bash\char`\"{} is called from the root of the catkin workspace
\item 3. roscore is called in an additional terminal
\item 4. Finally run {\ttfamily rosrun \hyperlink{namespaceci__example__python}{ci\+\_\+example\+\_\+python} \hyperlink{demo__ros__pid_8py}{demo\+\_\+ros\+\_\+pid.\+py}}
\end{DoxyItemize}

Notice the use of the double \#\# for the comments. This allow Doxygen to parse you code and for you to explain in details the content of the demo.


\begin{DoxyCodeInclude}
1 \textcolor{comment}{#!/usr/bin/env python}
2 
3 
4 \textcolor{stringliteral}{""" @namespace Demos of the ci\_example\_python.pid.PID controller using ROS param.}
5 \textcolor{stringliteral}{}
6 \textcolor{stringliteral}{@file demo\_ros\_pid.py}
7 \textcolor{stringliteral}{@copyright Copyright (c) 2017-2019,}
8 \textcolor{stringliteral}{           New York University and Max Planck Gesellschaft,}
9 \textcolor{stringliteral}{           License BSD-3-Clause}
10 \textcolor{stringliteral}{}
11 \textcolor{stringliteral}{@example demo\_ros\_pid.py}
12 \textcolor{stringliteral}{In order to run this demos the `ci\_example\_python/python/ci\_example\_python` path}
13 \textcolor{stringliteral}{needs to be added in the `PYTHONPATH`. This is done after:}
14 \textcolor{stringliteral}{- 1. "catkin\_make" is called from the root of the catkin workspace}
15 \textcolor{stringliteral}{- 2. "source ./devel/setup.bash" is called from the root of the catkin workspace}
16 \textcolor{stringliteral}{- 3. roscore is called in an additional terminal}
17 \textcolor{stringliteral}{- 4. Finally run `rosrun ci\_example\_python demo\_ros\_pid.py`}
18 \textcolor{stringliteral}{}
19 \textcolor{stringliteral}{Notice the use of the double ## for the comments. This allow Doxygen}
20 \textcolor{stringliteral}{to parse you code and for you to explain in details the content of the demo.}
21 \textcolor{stringliteral}{"""}
22 
23 
24 \textcolor{comment}{# Python 3 compatibility, has to be called just after the hashbang.}
25 \textcolor{keyword}{from} \_\_future\_\_ \textcolor{keyword}{import} print\_function, division
26 \textcolor{keyword}{import} rospy
27 \textcolor{keyword}{from} \hyperlink{namespaceci__example__python_1_1pid}{ci\_example\_python.pid} \textcolor{keyword}{import} get\_ros\_params\_pid
28 
29 
30 \textcolor{keywordflow}{if} \_\_name\_\_ == \textcolor{stringliteral}{"\_\_main\_\_"}:
31     
32     \textcolor{comment}{# here we set the parameters to ROS.}
33     rospy.set\_param(\textcolor{stringliteral}{'kp'}, 1.0)
34     rospy.set\_param(\textcolor{stringliteral}{'kd'}, 1.0)
35     rospy.set\_param(\textcolor{stringliteral}{'ki'}, 1.0)
36 
37     
38     current\_position = 1
39     
40     current\_velocity = 0.1
41     
42     position\_target = 0
43     
44     delta\_time = 0.001
45 
46     \textcolor{keywordflow}{print} (\textcolor{stringliteral}{"pid using ros parameter server"})
47 
48     
49     pid = \hyperlink{namespaceci__example__python_1_1pid_ae1f8a5e73e269ede0bdd4d0daf2f265e}{get\_ros\_params\_pid}()
50     \textcolor{keywordflow}{print} (pid)
51 
52     
53     control = pid.compute(current\_position,current\_velocity,position\_target,delta\_time)
54     \textcolor{keywordflow}{print} (\textcolor{stringliteral}{"control: "},control)
\end{DoxyCodeInclude}
 