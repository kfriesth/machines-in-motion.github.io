\subsection*{I. Introduction}

These are general rules independent of the coding language.

\subsection*{II. Catkin Package and Repository Naming}


\begin{DoxyItemize}
\item Name\+: lower\+\_\+cases\+\_\+with\+\_\+underscore.
\item One repository per catkin package.
\item Same name for package and repository.
\end{DoxyItemize}

\subsection*{I\+II. Versioning}

Use semantic versioning when versioning packages. Short summary\+:


\begin{DoxyItemize}
\item Given a version number M\+A\+J\+O\+R.\+M\+I\+N\+O\+R.\+P\+A\+T\+CH, increment the\+:
\item M\+A\+J\+OR version when you make incompatible A\+PI changes,
\item M\+I\+N\+OR version when you add functionality in a backwards-\/compatible manner, and
\item P\+A\+T\+CH version when you make backwards-\/compatible bug fixes.
\end{DoxyItemize}

Regarding formatting of pre-\/releases (alpha, beta, rc, ...) consider P\+E\+P440, especially when versioning Python packages (note that there is no general conflict with semantic versioning, only the format is a bit different).

\subsection*{IV. Contribution Guidelines}

\subsubsection*{I\+V.\+1. Code revision\+:}

To ensure some level of code quality, all modifications have to be reviewed before they can be merged. This is done via merge requests in Git\+Lab (on Git\+Hub it is called pull request but the concept is the same).

The procedure for adding a change is as follows\+:


\begin{DoxyItemize}
\item When creating branches, use your name as a namespace, e.\+g. {\itshape rickdeckard/my\+\_\+branch}. This is to avoid confusion with branches of other people and to indicate who is responsible for that branch. It is not allowed to push branches without such namespace.
\item Test all your changes and update documentation.
\item When finished, create a merge request to the master branch of the main repository.
\item Add one or more of the maintainers as reviewers. The reviewers will review and maybe request changes.
\item Merging is done by the contributor but is only allowed after all reviewers approved. If the merge will affect other people (e.\+g. by breaking the A\+PI), make sure to inform them, e.\+g. via a Mail on an appropriate mailing list. After a merge request is merged, the feature branch shall be deleted.
\end{DoxyItemize}

Never push directly to master or other top-\/level branches. Never force-\/push to any branch that others are using as well. All your development should happen in branches inside your namespace. To keep the number of branches on the repository low, make sure to delete branches that are not needed anymore.

The top-\/level namespace of the repository should only contain a master branch and maybe a few branches for specific versions. Those branches should be protected so that direct pushing is not possible (everything has to go through merge requests).

\subsubsection*{I\+V.\+2. Some rules for the contributors\+:}


\begin{DoxyItemize}
\item To make life easier for the reviewers and to get your changes merged quickly (thus reducing the risk of merge conflicts), try to keep merge requests rather small. This means that, where possible, a bigger task should be split into smaller sub-\/tasks that can be merged one after another instead of putting everything in one huge merge request.
\item When synchronising your feature branch with upstream, you may want to prefer git rebase over git merge to keep the history of you branch clean. However, when there are complicated merge conflicts, it can be much less painful to use merge, which is okay in this case.
\item On your own branches, you can do whatever you want, i.\+e. it is usually okay to force push there (and even necessary when you rebase). However, when applying changes requested by a reviewer, please do not amend or squash them into older commits but add them as new commits. This makes it easier for the reviewer to see what changed.
\item Do not add functional changes and major reformatting in the same commit as this makes review of the functional changes very difficult.
\item Do add unit tests for new features
\item Do create new demos or update existing demos to make it easier how to use the updated A\+PI
\end{DoxyItemize}

\subsubsection*{I\+V.\+3. Some rules for the reviewers\+:}


\begin{DoxyItemize}
\item To make life easier for the contributors, try to provide reviews in a timely manner. The contributor may be blocked in continuing their work while the merge request is under review.
\item When reviewing, check the following things\+:
\begin{DoxyItemize}
\item Is the style guide followed?
\item Are new features/changes properly documented?
\item And of course\+: Do the changes look reasonable and correct?
\end{DoxyItemize}
\item The one who merges should directly delete the feature branch afterwards.
\end{DoxyItemize}

\subsection*{V. Documentation}

\subsubsection*{V.\+1 In-\/code documentation}

All code {\bfseries shall} be documented in-\/source using Doxygen. This means that every function, class, struct, global variable, etc, needs to have a docstring containing some documentation in\+:
\begin{DoxyItemize}
\item \href{https://google.github.io/styleguide/pyguide.html?showone=Comments#Comments}{\tt Google} format for Python,
\item \href{http://www.doxygen.nl/manual/index.html}{\tt Doxygen} format for C/\+C++.
\end{DoxyItemize}

If the definition and declaration are separated (C/\+C++), Doxygen should be in the header, not the cpp file. {\bfseries Please do not duplicate!}.

If you are using catkin, the package should include the mpi\+\_\+cmake\+\_\+modules package and in the C\+Make\+Lists.\+txt file call the {\ttfamily build\+\_\+doxygen\+\_\+documentation()} or the {\ttfamily default\+\_\+mpi\+\_\+cmake\+\_\+modules()}(future) macros. The documentation can then be build via catkin\+\_\+make -\/\+D\+B\+U\+I\+L\+D\+\_\+\+D\+O\+C\+U\+M\+E\+N\+T\+A\+T\+I\+ON=ON.

Make the documentation as compact as possible. Avoid boilerplate formulations that do not add useful information, e.\+g. instead of \char`\"{}\+This function returns foobar\char`\"{} simply write \char`\"{}\+Returns foobar\char`\"{}.

Also add regular comments to the code whenever you feel that they would help a future reader to more easily understand what that code is doing.

\subsubsection*{V.\+2 Unit-\/tests and demos}

Please do {\bfseries N\+OT} neglect the power of the continuous intergation and a nice written demos in terms of Documentation. So {\bfseries please} take the time to\+:
\begin{DoxyItemize}
\item write some unit-\/tests. See how to write a unit-\/tests from the tests folder in this package.
\item write demo executables to make the external user understand how your A\+PI should be used. 
\end{DoxyItemize}