\subsection*{1. Introduction}

This module manages the build of the documentation. The input files are C++, Python, and Markdown. In order to process this we use a couple of off-\/the-\/shelf softwares (see the list below).


\begin{DoxyItemize}
\item Doxygen the C++ api documentation parser,
\item Breathe a sphinx extension that parse the doxygen xml output into restructured text files,
\item recommonmark a sphinx extension parsing markdown files.
\item sphinx-\/apidoc the Python api documentation parser,
\item Sphinx the documentation renderer,
\end{DoxyItemize}

If anything seems fuzzy (it is a rather long and tedious pipeline), please let me know via posting an issue \href{https://github.com/machines-in-motion/mpi_cmake_modules/issues}{\tt here}.

\subsection*{2. Advanced explanation on the tools}

In order to build the documentation we need to setup the following tools\+:
\begin{DoxyItemize}
\item \href{http://www.doxygen.nl/}{\tt Doxygen} the C++ api documentation parser,
\item \href{https://breathe.readthedocs.io/en/latest/}{\tt Breathe} a sphinx extension that parse the doxygen xml output into restructured text files,
\item \href{https://recommonmark.readthedocs.io/en/latest/}{\tt recommonmark} a sphinx extension parsing markdown files.
\item \href{http://www.sphinx-doc.org/en/master/man/sphinx-apidoc.html}{\tt sphinx-\/apidoc} the Python api documentation parser,
\item \href{http://www.sphinx-doc.org/en/master/}{\tt Sphinx} the documentation renderer,
\end{DoxyItemize}

\subsubsection*{2.\+1 Doxygen}

In order to execute to generate the C++ A\+PI documentation we use the Doxygen tool. We wrote a {\ttfamily Doxyfile}, used to parameter Doxyygen, to notably\+:


\begin{DoxyItemize}
\item Output the files in the {\ttfamily \+\_\+build/docs/doxygen} folder with the {\ttfamily O\+U\+T\+P\+U\+T\+\_\+\+D\+I\+R\+E\+C\+T\+O\+RY} parameter.
\item Generate a list of xml files containing the A\+PI documentation setting the {\ttfamily G\+E\+N\+E\+R\+A\+T\+E\+\_\+\+X\+ML} to {\ttfamily Y\+ES}.
\end{DoxyItemize}

The Makefile looks at the {\ttfamily Doxyfile} in {\ttfamily doc\+\_\+config\+\_\+files/doxygen/} and C\+Make configure the {\ttfamily Doxyfile.\+in} from {\ttfamily cmake/doxygen/}.

\subsubsection*{2.\+2 Breathe}

This tool is a module of sphinx that parse the Doxygen xml output. Breathe provide two import tools\+:


\begin{DoxyItemize}
\item An A\+PI that allow you to reference to the object from the Doxygen xml output.
\item An executable {\ttfamily breathe-\/apidoc} that generates automatically the C++ A\+PI into Re\+Structed files.
\end{DoxyItemize}

In order to use it we need to add a couple of line in the {\ttfamily conf.\+py} used by Sphinx\+:


\begin{DoxyCode}
extensions = [
    # ... other stuff
    'breathe', # to define the C++ api
    # ... other stuff
]
\end{DoxyCode}


We also need to add the following variable that determine the behavior of Breathe\+:


\begin{DoxyCode}
# breath project names and paths. Here project is the name of the repos and the path is the path to the
       Doxygen output.
breathe\_projects = \{ project: "../doxygen/xml" \}
# Default project used for all Doxygen output (we use only one here).
breathe\_default\_project = project
# By default we ask all informations to be displayed.
breathe\_default\_members = ('members', 'private-members', 'undoc-members')
\end{DoxyCode}


Once the {\ttfamily conf.\+py} is setup we execute {\ttfamily breath-\/apidoc} on the Doxygen xml output\+: \begin{DoxyVerb}breathe-apidoc -o $(BREATHE_OUT) $(BREATHE_IN) $(BREATHE_OPTION)
\end{DoxyVerb}


with\+:


\begin{DoxyItemize}
\item {\ttfamily B\+R\+E\+A\+T\+H\+E\+\_\+\+O\+UT} the output path ({\ttfamily \+\_\+build/docs/sphinx/breathe/}),
\item {\ttfamily B\+R\+E\+A\+T\+H\+E\+\_\+\+IN} the path to the Doxygen xml output ({\ttfamily \+\_\+build/docs/doxygen/xml/}),
\item and {\ttfamily B\+R\+E\+A\+T\+H\+E\+\_\+\+O\+P\+T\+I\+ON} some output formatting option, here empty.
\end{DoxyItemize}

This breathe-\/apidoc will generate the list of all classes, namespace and files in a different Re\+Structured\+Text ({\ttfamily .rst}) files. We will use them to generate the final layout of the documentation.

\subsubsection*{2.\+3 recommonmark}

We want to have an easy and intuitive way of writing extra documentation from the code. Hence our choice to use {\ttfamily Markdown} and {\ttfamily Restructured} text. Both file types are parse by sphinx and converted to html. The sphinx module {\ttfamily recommonmark} is here to convert the Markdown properly.

In the header of the {\ttfamily conf.\+py} used by sphinx we need to include\+: \begin{DoxyVerb}# AutoStructify for math in markdown
import recommonmark 
from recommonmark.transform import AutoStructify
\end{DoxyVerb}


We add it to the {\ttfamily extensions} variable in the same file\+:


\begin{DoxyCode}
extensions = [
    # ... other stuff
    'recommonmark', # markdown support
    # ... other stuff
]
\end{DoxyCode}


Then we tell Sphinx to read the .md extension files in the {\ttfamily conf.\+py}\+:


\begin{DoxyCode}
# The suffix(es) of source filenames.
source\_suffix = ['.rst', '.md']
\end{DoxyCode}


And last in order to get math support in the mardown using\+: \begin{DoxyVerb}```math
    a = \theta
```
\end{DoxyVerb}


We need to add the following at the end of the {\ttfamily conf.\+py}\+:


\begin{DoxyCode}
# some tools for markdown parsing
def setup(app):
app.add\_config\_value('recommonmark\_config', \{
        'auto\_toc\_tree\_section': 'Contents',
        'enable\_math':True,
        'enable\_inline\_math':True,
        'enable\_eval\_rst':True,
        \}, True)
app.add\_transform(AutoStructify)
\end{DoxyCode}


\subsubsection*{2.\+4 sphinx-\/apidoc}

This tool allow the generation of a Python module A\+PI documentation extracting the doc string from the code. We need to add to the P\+Y\+T\+H\+O\+N\+P\+A\+TH the path to the Python module in the {\ttfamily conf.\+py}\+:


\begin{DoxyCode}
sys.path.insert(0, os.path.abspath("path/to/the/python/module"))
\end{DoxyCode}


add the according Sphinx extensions\+:


\begin{DoxyCode}
extensions = [
    # ... other
    'sphinx.ext.autodoc',
    'sphinx.ext.doctest',
    'sphinx.ext.intersphinx',
    'sphinx.ext.todo',
    'sphinx.ext.coverage',
    'sphinx.ext.mathjax',
    'sphinx.ext.ifconfig',
    'sphinx.ext.viewcode',
    'sphinx.ext.githubpages',
    # ... other
]
\end{DoxyCode}


And then build the A\+PI documentation by\+: \begin{DoxyVerb}sphinx_apidoc -o $(SPHINX_BUILD_OUT) path/to/the/python/module
\end{DoxyVerb}


Where {\ttfamily S\+P\+H\+I\+N\+X\+\_\+\+B\+U\+I\+L\+D\+\_\+\+O\+UT} is the output path.

\subsubsection*{2.\+5 sphinx-\/build}

The final layout is managed here and build using {\ttfamily shpinx-\/build}. The tricky thing with {\ttfamily sphinx-\/build} is that everything included needs to be in the working directory. Therefore in the build directory we set the output of {\ttfamily breathe-\/apidoc} and {\ttfamily shpinx-\/apidoc} to {\ttfamily \+\_\+build/docs/sphinx}. And inside the same folder we create a symlink that points to the source {\ttfamily doc/} folder.

Therefore in order\+:


\begin{DoxyItemize}
\item The {\ttfamily index.\+rst} includes the C++ A\+PI main {\ttfamily .rst} files from Breath.
\item Then it includes the {\ttfamily modules.\+rst} file from {\ttfamily sphinx-\/apidoc}
\item And then is adds all files inside {\ttfamily doc/}, which, again, points toward the source {\ttfamily doc/} directory.
\end{DoxyItemize}

The command to execute is the following\+: \begin{DoxyVerb}sphinx-build -M html _build/docs/sphinx _build/docs/sphinx
\end{DoxyVerb}


This will generate the documentation website in {\ttfamily \+\_\+build/docs/sphinx/html/} Thefore {\ttfamily firefox \+\_\+build/docs/sphinx/html/index.\+html} opens the documentation

\subsection*{3. Feedback on all this}

If anything seems fuzzy (it is a rather long and tedious pipeline), please let me know via posting an issue \href{https://github.com/machines-in-motion/mpi_cmake_modules/issues}{\tt here}. 