\subsection*{Introduction}

This package provide some tools in order to format the C/\+C++ code using clang-\/format and the \href{https://machines-in-motion.github.io/code_documentation/ci_example_cpp/coding_guidelines_1.html}{\tt machines-\/in-\/motions} specific set of rules.

\subsection*{Executable}

In order to use it one need to source the workspace environment\+: \begin{DoxyVerb}source workspace/devel/setup.bash
\end{DoxyVerb}


And to run the following command\+: \begin{DoxyVerb}rosrun mpi_cmake_modules clang_format <list of files> <list of folders>
\end{DoxyVerb}


$<$list\+\_\+of\+\_\+files$>$ and 
\begin{DoxyItemize}
\end{DoxyItemize}be either relative paths or absolute paths.

For those not willing to use rosrun the executable is located in \begin{DoxyVerb}mpi_cmake_modules/scripts/clang_format
\end{DoxyVerb}


So one can simply get the full path of the executable in order to use it. A cleaner way would be to properly install the executable script.

The executable will create the list of all files to be formatted by checking all arguments (which order does not matter). And perform the following tests\+:
\begin{DoxyItemize}
\item If you provided a file it will keep it if\+:
\begin{DoxyItemize}
\item it exists \&
\item it is a source files
\end{DoxyItemize}
\item If you provided a folder it will search recursively all the files and perform the above checks.
\end{DoxyItemize}

Once the list is created the tool format each selected files.

\subsection*{C\+Make macro}

This package also provide a C\+Make macro allowing you to perform automatically the formatting upon build.

The macro to add is located in the {\ttfamily mpi\+\_\+cmake\+\_\+modules/cmake/clang-\/format.\+cmake} and is called \begin{DoxyVerb}format_code()
\end{DoxyVerb}


By default it does nothing. In order to activate it you need to add the folowwing C\+Make argument\+: \begin{DoxyVerb}catkin build mpi_cmake_modules --cmake-args -DFORMAT_CODE=ON \end{DoxyVerb}
 